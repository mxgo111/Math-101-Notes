\section{Lecture 1: 01/25/2021}

\subsection{The Raisin Game}

You and your friend start with $n$ raisins, where $n$ is a positive integer (e.g. $1, 2, \dots$). You decide to play a game with your friend where you each take $1, 2, 3,$ or $4$ raisins when it is your turn. The person who takes the last raisin wins. \\

How do we come up with a strategy for this game? After playing a few games, you might realize that the number $5$ is quite special. If your friend reaches the number $5$ (and then it's your turn), then unfortunately, no matter how many raisins you take, your friend can take the last raisin by taking the remaining raisins.\\

After playing some more, we realize that for the numbers $6, 7, 8,$ and $9$, whoever goes first wins! The first player can force the next player to start with $5$ raisins, which results in a win for the first player.\\

At this point, we might form the following conjecture:

\begin{conjecture}
The second player has a winning strategy iff (if and only if, or ``exactly when") at the start the number of raisins is a multiple of $5$.
\end{conjecture}

\begin{proof}
First we do the ``if" direction. Suppose the number of raisins is a multiple of $5$. Then the first player takes some number of raisins $x$, where $x$ is $1, 2, 3,$ or $4$. Then the second player can take $5-x$ raisins, and the remaining number of raisins is another multiple of $5$. Thus, the second player can continue to force the first player to start at smaller and smaller multiples of $5$ until they reach the number $5$. The first player then takes some number of raisins, and as discussed above the second player wins no matter how many the first player takes.\\

Next we do the ``only if" direction. Suppose the number of raisins is not a multiple of $5$. Then the first player can win by taking a number of raisins that leaves a multiple of $5$, so the argument above shows you have a winning strategy.
\end{proof}

Here is an example of some conjectures that are not always true. The mathematician Leonhard Euler stated that:
\begin{conjecture}
$n^2 + n + 41$ is prime for $n = 0, \dots, 39$.
\end{conjecture}
This might lead one to think that the above is true for all $n\geq0$, but we can provide a counterexamples for things outside this range such as $40^2 + 40 + 41$. (You will learn about techniques to prove that for $n>39$ Euler's polynomial is not prime later in the course).
Here is another from Fermat
\begin{conjecture}
$2^{2^n} + 1$ is prime for $n = 0, 1, 2,3,4$ and composite for $n=5$.
\end{conjecture}


\subsection{Fundamentals}

Some important definitions!

\begin{definition}[Evenness]
An integer $n$ is \textit{even} if $n = 2k$ for some integer $k$.
\end{definition}

These ``if"'s should actually be ``iff"'s (if and only if). However, in definitions, we usually assume that if means iff.

\begin{definition}[Oddness]
Correspondingly, an integer $n$ is \textit{odd} if $n = 2k+1$ for some integer $k$. 
\end{definition}

\begin{definition}[Divisibility]
An integer $m$ \textit{divides} an integer $n$, written $m|n$, if $n = mk$ for some integer $k$.
\end{definition}

\begin{definition}[Casework]
Sometimes when we write a proof, we want to split up the possibilities to prove it in ``cases".
\end{definition}

\begin{example}
Prove that $n(3n + 1)$ is always even. 
\end{example}

\begin{proof}
We begin with a generality statement. This means we pick an arbitrary integer $n$ and then proceed with our proof. Because we just picked an arbitrary number, our proof will still hold. We can split this into two cases: when $n$ is even, when $n$ is odd.\\

\textbf{Case 1}: $n$ is even.\\

If $n$ is even we have that $n = 2k$ for some integer k, so 
\[
n(3n +1) = 2k(3n + 1),
\]
which is even.\\

\textbf{Case 2}: $n$ is odd.\\

If $n$ is odd, we can write that $n = 2k+1$ for some integer $k$, so 
\[
n(3n+1) = n(6k+4) = 2(n(3k+2)),
\]
which is even.
\end{proof}

