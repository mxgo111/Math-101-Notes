\section{Lecture 12: 03/15/21}

\subsection{Operation on Cosets, Normal Subgroups}

Let $H$ be a subgroup of $G$.

\begin{lemma}
Let $g_1, g_2 \in G$. Then $g_1H = g_2H$ iff $g_1 \in g_2H$. This means that $g_1 = g_2 h$ for some $h \in H$.
\end{lemma}

\begin{definition}
We defined $*_l$ on $\{\text{left cosets of $H$ in $G$}\}$ by:
\begin{align*}
    (aH) *_l (bH) := abH.
\end{align*}
\end{definition}

When is $*_l$ well-defined?\\

We want it to be the case that if $a_1H = a_2H$ an $b_1H = b_2H$, then $a_1b_1H = a_2b_2H$.

We know that $a_1 = a_2h_1$ and $b_1 = b_2h_2$ for some $h_1, h_2 \in H$. We want to show that $a_1b_1 = a_2b_2h_3$ for some $h_3 \in H$.

Thus, our expression that we want to show is equivalent to:
\begin{align*}
    a_1b_1 &= a_2b_2h_3 \\
    \iff (a_2h_1)(b_2h_2) &= a_2b_2h_3 \\
    \iff b_2^{-1}h_1b_2h_2 &= h_3
\end{align*}

Thus, we need $b_2^{-1}h_1b_2$ to be in $H$. We write $g = b_2^{-1}$ and $h = h_1$, so that we now need $ghg^{-1} \in H$ for all $g \in G$ and $h \in H$, which is equivalent to $H$ being a normal subgroup of $G$ (denoted $H \triangleleft G$).

\subsection{Quotient Groups}

\begin{proposition}
Suppose that $H \triangleleft G$. We write $G/H$ for the set of left cosets of $H$ in $G$. We've defined an operation on $G/H$ by $(aH)(bH) := abH$, and we've seen that this operation is well-defined. Prove that $G/H$ is a group under this operation.\\

We read $G/H$ as ``$G$ mod $H$" and call it a \textbf{quotient group} or \textit{factor group}.
\end{proposition}

We can check the axioms as follows:
\begin{itemize}
    \item Closure: If $aH, bH \in G/H$, then $(aH)(bH) = abH$ is a left coset, so $abH \in G/H$.
    \item Associativity: Let $aH$, $bH$, $cH$ be in $G/H$. Then $((aH)(bH)(cH) = ((ab)cH) = (a(bc)H) = (aH)((bH)(cH))$ (by associativity in $G$).
    \item Identity: Let $e$ be the identity of $G$. Then $(aH)(eH) = (eH)(aH) = aH$.
    \item Inverses: Let $g \in G$. Then $g^{-1}H$ is the inverse of $gH$ since $(g^{-1}H)(gH) = (gH)(g^{-1}H) = eH$.
\end{itemize}

\begin{example}
Let $G$ be any group. We examine edge cases. We can show that $G$ and $\{e\}$ are normal subgroups of $G$.

\begin{itemize}
    \item Describe the quotient group $G/G$.
    \begin{align*}
        G/G &= \{\text{left cosets of $G$ in $G$}\} \\
        &= \{eG\}.
    \end{align*}
    so this quotient group has just one element!
    \item Describe the quotient group $G/\{e\}$.
    \begin{align*}
        G/\{e\} &= \{\text{left cosets of $\{e\}$ in $G$}\} \\
        &= \{\{g\} \ \forall g \in G\}
    \end{align*}
    so $|G/\{e\}| = |G|$.
\end{itemize}
\end{example}

\begin{example}
Explain why $\Z \triangleleft \R$, and describe the quotient group $\R / \Z$.
\end{example}

\begin{proof}
Two different ways to show that $\Z$ is a normal subgroup of $\R$:
\begin{itemize}
    \item $\R$ is abelian
    \item If $g \in \R$ and $h \in \Z$, then $g + h + (-g) = h \in \Z$.
\end{itemize}
The left cosets are $a + \Z$, so we have:
\[
\R/\Z = \{a + \Z | 0 \le a < 1\}
\]
\end{proof}

\begin{example}
Prove that, if $H \triangleleft G$ and $a \in G$, then $aH = Ha$.
\end{example}

\begin{proof}
Suppose $H \triangleleft G$ and $a \in G$. Then we show both sides:
\begin{itemize}
    \item $\subseteq$. Let $x \in aH$. Then $x = ah$ for some $h \in H$. We can rewrite $x = aha^{-1}a$. Since $H$ is a normal subgroup of $G$, $aha^{-1} \in H$, so $x \in Ha$.
    \item $\supseteq$. Let $x \in Ha$. Then $x = ha$ for some $h \in H$. We can rewrite $x = aa^{-1}ha$, and $a^{-1}ha \in H$, so $x \in aH$.
\end{itemize}
\end{proof}