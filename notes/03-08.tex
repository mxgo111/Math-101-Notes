\section{Lecture 10: 03/08/21}

\subsection{Pigeonhole Principle}

\begin{definition}[Pigeonhole Principle]
If you put $n+1$ pigeons in $n$ holes, then there is a hole with more than $1$ pigeon.
\end{definition}

\begin{example}
Let $G$ be a finite group, and let $g\in G$. Prove that there exists a positive integer $n$ such that $g^n=e$.
\end{example}

\begin{proof}
Consider $g,g^2,g^3,\dots\in G$. Because we have an infinite list within a finite group, by the Pigeonhole Principle there exists $j, k \in \N$ with $j > k$ such that $g^j = g^k$. ``Multiplying'' both sides by $g^{-k}$ on the right, 
\[g^jg^{-k}=g^kg^{-k}\]
or \[g^{j-k}=e\]
\end{proof}

\subsection{Order}

\begin{definition}[Order]
Let $G$ be a group (finite or infinite), and let $g\in G$. If there exists $n\in \N$ such that $g^n=e$, the smallest such $n$ is called the \textit{order} of $g$. The order of $g$ is denoted by $|g|$ or $o(g)$.
\end{definition}

Recall that we once talked about the order of a \textit{group}, which was defined to be the number of elements in the group, also denoted by $|G|$. Now, however, we are talking about the order of an \textit{element} of a group.

\begin{example}
Find the order of $3$ in $U_{13}$.
\end{example}

\begin{proof}
Note that $3^1 = 3$, $3^2 = 9$, and $3^3 = 27$, but $[27]_{13} = [1]_{13}$, so the order is $3$.
\end{proof}

\begin{example}
Find the order of $5$ in $\Z_{20}$.
\end{example}

\textbf{Solution.} 
In $\Z_{20}$, the operation is addition, so raising $5$ to a power is like adding it to itself over and over. 
\[5=5\]
\[5+5=10\]
\[5+5+5=15\]
\[5+5+5+5=20=0\]
So the order of $5$ is $4$.

\begin{example}
Prove the following
\begin{proposition}
Let $G$ be a group. Suppose $a\in G$ has finite order. If $a^k=e$ for some $k\in \Z$, then $|a| \big| k$
\end{proposition}
\end{example}
\begin{proof}
Let $n=|a|$. (WTS: $n|k$.) By the division algorithm, there exists $q,r\in \Z$ such that $k=nq+r$ and $0\leq r<n$. (WTS: $r=0$.) Then, $r=k-nq$, and \[a^r=a^{k-nq}=a^ka^{-nq}=a^k(a^n)^{-q}\] by exponent rules. Then \[a^r=(a^n)^{-q}=e\] since $a^k=e$ and $n=|a|$, so $a^n=e$. 
\medskip

We've shown that $a^r=e$ and $r<n$, but $n$ is the smallest positive integer such that $a^n=e$ so $r$ must not be a positive integer. Since $r\geq 0$, the only possibility is that $r=0$.
\end{proof}

\subsection{Generators, Cyclic Groups}

\begin{definition}
Let $G$ be a group.
\begin{itemize}
    \item If $a\in G$, then the \textit{subgroup generated by $a$} is 
    \[\langle a \rangle := \{a^k|k\in \Z\}\]
    \item $G$ is \textit{cyclic} if $G=\langle a \rangle$ for some $a\in G$. We call $a$ the \textit{generator} of $G$.
\end{itemize}
\end{definition}

\begin{example}
Is $S_3$ cyclic? Why or why not?
\end{example}
\begin{proof}[Solution.] None of the below subgroups generated by elements of $S_3$ are equal to $S_3$:
\[\langle e \rangle = \{e\}\]
\[\langle (1\ 2) \rangle = \{(1\ 2), e\}\]
\[\langle (1\ 3) \rangle = \{(1\ 3), e\}\]
\[\langle (2\ 3) \rangle = \{(2\ 3), e\}\]
\[\langle (1\ 2\ 3) \rangle = \{(1\ 2\ 3), (1\ 3\ 2)e\}\]
\[\langle (1\ 3\ 2)\rangle = \{(1\ 2\ 3), (1\ 3\ 2)e\}\]
So $S_3$ is not cyclic.
\end{proof}

\begin{example}
If $G$ is a finite cyclic group with a generator $a$, how does $|G|$ relate to $|a|$?
\end{example}
\textbf{Solution.} We claim that $|a|=|G|$.
\begin{proof}[Sketch of the proof:]
Let $n=|a|$. 
\begin{itemize}
    \item Prove that $G=\{e,a,a^2,\dots,a^{n-1}\}$
    \begin{itemize}
        \item $\langle a \rangle \subseteq G$ by definition of $\langle a \rangle$
        \item $\langle a \rangle \supseteq G$: $G= \{a^k|k\in \Z\}$ so WTS: $a^k\in \{e,a,a^2,\dots,a^{n-1}\}$. Use the Division Algorithm to write $k=nq+r$. Show that $a^k=a^r$.
    \end{itemize}
    \item Prove that $e,a,a^2,\dots,a^{n-1}$ are distinct.
\end{itemize}
\end{proof}

\begin{example}
Prove or disprove: Every cyclic group is abelian.
\end{example}
\begin{proof}
Let $G$ be a cyclic group with a generator $a$. Let $x,y\in G$. Then $x =a^j$ and $y=a^k$ for some $j,k\in \Z$. Then $xy=a^ja^k=a^{j+k}$ and $yx=a^ka^j=a^{k+j}$, so $xy-yx$ since $j+k=k+j$.
\end{proof}